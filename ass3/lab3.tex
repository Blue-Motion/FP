\documentclass[a4paper,11pt]{article}
\usepackage[utf8]{inputenc}
\usepackage{amsmath}
\usepackage{amssymb}
\usepackage[dutch]{babel}
\usepackage{color}
\usepackage{enumerate}
\usepackage{float}
\usepackage[margin=3cm]{geometry}
\usepackage{graphicx}
%\usepackage[hidelinks]{hyperref}
\usepackage{hyperref}
\usepackage{listings}
%\usepackage{subfig}
\usepackage{url}
%\usepackage[square]{natbib}

\definecolor{Gray}{gray}{0.9}
 
\lstset{language=Haskell,backgroundcolor=\color{Gray},basicstyle=\footnotesize,numberstyle=\footnotesize,numbers=left,stepnumber=1,numbersep=5pt,breaklines=true,frame=lines,tabsize=2}
 
\author{Thom Carretero Seinhorst (s1898760) \and Bart Offereins (s2255243)}
\date{\today}
\title{Functional programming \\Lab 3}

\begin{document}

  \maketitle

\section{Integer Expressions}

\subsection{}
\begin{lstlisting}
instance Show Expr where
  show (Val a) = show a
  show (Var a) = show a
  show (a :+: b) = "(" ++ show a ++ " + " ++ show b ++ ")"
  show (a :-: b) = "(" ++ show a ++ " - " ++ show b ++ ")"
  show (a :*: b) = show a ++ " * " ++ show b
  show (a :/: b) = show a ++ " / " ++ show b
  show (a :%: b) = show a ++ " % " ++ show b
\end{lstlisting}

\subsection{}
The nub function removes duplicate elements from a list. In particular, it keeps only the first occurrence of each element. For sorting the list we used of course the function sort. 
\begin{lstlisting}
vars :: Expr -> [Name]
vars = nub . sort . var
  where
  var (Val a) = []
  var (Var a) = [a]

  var (a :+: b) = var a ++ var b
  var (a :-: b) = var a ++ var b
  var (a :*: b) = var a ++ var b
  var (a :/: b) = var a ++ var b
  var (a :%: b) = var a ++ var b
\end{lstlisting}

\subsection{}
\begin{lstlisting}
valuations :: [(Name,Domain)] -> [Valuation]
valuations [] = []
valuations ((x,y):[]) =  [[(x,v)] | v <- y]
valuations (x:xs) = concat [(map (y++) (valuations xs)) | y <- (valuations [x])]
\end{lstlisting}

\subsection{}
\begin{lstlisting}
evalExpr :: Expr -> Valuation -> Integer
evalExpr e v
  | ((length . vars) e) > (length v) = error "not enough valuated parameters"
  | otherwise = eval e v
    where eval (Var x) v = head [snd n | n <- v, (fst n) == x] 
          eval (Val x) v = x
          eval (a :+: b) v = eval a v + eval b v
          eval (a :-: b) v = eval a v - eval b v
          eval (a :*: b) v = eval a v * eval b v
          eval (a :/: b) v = div (eval a v) (eval b v)
          eval (a :%: b) v = mod (eval a v) (eval b v)
\end{lstlisting}

\subsection{}
Given the lecture slides, it is quit easy to create our own parser. 

\begin{lstlisting}
lexer :: String -> [String]
lexer [] = []
lexer (c:cs)
  | elem c"+-*/%" = [c]:(lexer cs)
  | elem c " " = lexer cs
  | isAlpha c = (c:takeWhile isAlpha cs): lexer(dropWhile isAlpha cs)
  | isDigit c = (c:takeWhile isDigit cs): lexer(dropWhile isDigit cs)
  | otherwise = error "Syntax Error: invalid character in input"
  
parser :: String -> (Expr,[String])
parser str = parseE (lexer str)

parseE :: [String] -> (Expr,[String])
parseE tokens = parseE' acc rest
  where (acc, rest) = parseT tokens

parseT :: [String] -> (Expr,[String])
parseT tokens = parseT' acc rest
  where (acc, rest) = parseF tokens 

parseE' :: Expr -> [String] -> (Expr,[String])
parseE' accepted ("+":tokens) = parseE' (accepted :+: term) rest
  where (term, rest) = parseT tokens
parseE' accepted ("-":tokens) = parseE' (accepted :-: term) rest
  where (term, rest) = parseT tokens
parseE' accepted tokens = (accepted, tokens)

parseT' :: Expr -> [String] -> (Expr,[String])
parseT' accepted ("*":tokens) = parseT' (accepted :*: term) rest
  where (term, rest) = parseF tokens
parseT' accepted ("/":tokens) = parseT' (accepted :/: term) rest
   where (term, rest) = parseF tokens
parseT' accepted ("%":tokens) = parseT' (accepted :%: term) rest
   where (term, rest) = parseF tokens
parseT' accepted tokens =  (accepted, tokens)

parseF :: [String] -> (Expr, [String])
parseF [] = error "Parse error...abort"
parseF (tok:tokens)
  | tok == "("  = (expr, tail rest)
  | isAlpha (head tok) = (Var tok, tokens)
  | isDigit (head tok) = (Val (read tok), tokens)
  | otherwise = error ("Syntax Error: " ++ tok)
  where
    (expr, rest) = parseE tokens

toExpr :: String -> Expr
toExpr str = fst (parser str)
\end{lstlisting}

\subsection{}
We have put all related code in the moduel Expression. We exported the following functions to export: Expr, vars, evalExpr, toExpr. These functions are used in the other modules. The parse function is not used in other modules.

\subsection{Bonus}
\begin{lstlisting}
simplifyExpr :: Expr -> Expr
simplifyExpr ((Val a) :+: (Val b)) = Val (a + b)
simplifyExpr ((Val a) :-: (Val b)) = Val (a - b)
simplifyExpr ((Val a) :*: (Val b)) = Val (a * b)
simplifyExpr ((Val a) :/: (Val b)) = Val (div a b)
simplifyExpr ((Val a) :%: (Val b)) = Val (mod a b)
simplifyExpr (a :+: b) = (a + simplifyExpr b)
simplifyExpr (a :-: b) = (a - simplifyExpr b)
simplifyExpr (a :*: b) = (a * simplifyExpr b)
--simplifyExpr (a :/: b) = simplifyExpr (mod (simplifyExpr a) (simplifyExpr b))
--simplifyExpr (a :%: b) = simplifyExpr (mod (simplifyExpr a) (simplifyExpr b))
simplifyExpr a = a

\end{lstlisting}

\section{Solving a CSP}

A Constraint Satisfaction Problem (CSP) is a set constraints on variables,  where we seek the combination of variable values (within a domain) for which all equations in the set are true. In our case, all constraints are equations in the form of Expr Operator Expr. Represented in the input as a string. 

\subsection{toComparison}

To evaluate these constraints, we need to parse the constraint string. We already know how to parse an expression, so we only need to find a way to get the operator.

\begin{lstlisting}
comparisonOperator :: [Char] -> Relop
comparisonOperator "<" = LessThan
comparisonOperator "<=" = LessEqual
comparisonOperator "=" = Equal
comparisonOperator ">" = Greater
comparisonOperator ">=" = GreaterEqual
comparisonOperator "#" = NotEqual

expression :: [Char] -> [Char]
expression str = takeWhile (not.(\x -> elem x "<>=#")) str

getComparator:: [Char] -> [Char]
getComparator str = (reverse (dropWhile (not.(\x -> elem x "<>=#")) (reverse (dropWhile (not.(\x -> elem x "<>=#")) str))))

\end{lstlisting}

\begin{lstlisting}
toComparison :: String -> Comparison
toComparison str = Cmp (comparisonOperator (getComparator str)) (toExpr(expression str)) (toExpr((reverse . expression . reverse ) str))
\end{lstlisting}

\subsection{Evaluating Comparisons}
	
Now that we hve parsed the comparisons, we can evaluate them. This is done by pattern matching. We evaluate both expressions, filling in the Valuation for the different variables. Then, we match the provided operator and return whether the comparison, with this values is valid or not.
\begin{lstlisting}
evalCmp :: Comparison -> Valuation -> Bool
evalCmp (Cmp LessThan expression1 expression2) xs = (evalExpr expression1 xs) < (evalExpr expression2 xs)
evalCmp (Cmp LessEqual expression1 expression2) xs = (evalExpr expression1 xs) <= (evalExpr expression2 xs)
evalCmp (Cmp Equal expression1 expression2) xs 	= (evalExpr expression1 xs) == (evalExpr expression2 xs)
evalCmp (Cmp Greater expression1 expression2) xs = (evalExpr expression1 xs) > 	(evalExpr expression2 xs)
evalCmp (Cmp GreaterEqual expression1 expression2) xs= (evalExpr expression1 xs) >= (evalExpr expression2 xs)
evalCmp (Cmp NotEqual expression1 expression2) xs = (evalExpr expression1 xs) /= (evalExpr expression2 xs)
\end{lstlisting}
\subsection{solving a CSP}

By now, we can evaluate single comparisons, based on single values. However, the csp contains a textual representation of ranges for different variables, as well as multiple comprisons that should all be true. All this represented as a file, which we read as 1 String.

The first step (removing whitespace) is given. We proceed with separating domans string and the contraints string and give a tuple containting both as a result


\begin{lstlisting}
splitCSP :: String -> (String, String)
splitCSP s = ((takeWhile (/= '}') (tail (dropWhile (/= '{') s))) , (takeWhile (/= '}') (tail (dropWhile (/= '{') (tail (dropWhile (/= '{') s))))))

takeCSPitems :: (String,String) -> ([String],[String])
takeCSPitems (d,c) = (addItem d (numCSPitems d), addItem c (numCSPitems c))
  where addItem _ 0 = []
        addItem ds n = (takeWhile (/= ',') ds):(addItem (tail (dropWhile (/= ',') ds))) (n-1)
        numCSPitems ds = length (filter (== ',') ds) + 1

varDomain :: String -> (Name,Domain)
varDomain s = (takeWhile isAlpha s,[(read (takeWhile (isDigit) (dropWhile (not.isDigit) s)))..(read ((reverse . (takeWhile isDigit) . tail . reverse)  s ))])

parseCSPitem :: ([String],[String]) -> ([(Name,Domain)],[Comparison])
parseCSPitem (a,b) = (map varDomain a, map toComparison b)

solution :: ([(Name,Domain)],[Comparison]) -> [Valuation]
solution (v,c) = [x | x <- valuations v, (foldr (&&) True [evalCmp cmp x | cmp <- c]) == True]
\end{lstlisting}
\end{document}
